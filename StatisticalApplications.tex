%%%%%%%%%%%%%%%%%%%%%%%%%%%%%%%%%%%%%%%%%
% Stylish Article
% LaTeX Template
% Version 2.1 (1/10/15)
%
% This template has been downloaded from:
% http://www.LaTeXTemplates.com
%
% Original author:
% Mathias Legrand (legrand.mathias@gmail.com) 
% With extensive modifications by:
% Vel (vel@latextemplates.com)
%
% License:
% CC BY-NC-SA 3.0 (http://creativecommons.org/licenses/by-nc-sa/3.0/)
%1
%%%%%%%%%%%%%%%%%%%%%%%%%%%%%%%%%%%%%%%%%

%----------------------------------------------------------------------------------------
%	PACKAGES AND OTHER DOCUMENT CONFIGURATIONS
%----------------------------------------------------------------------------------------

\documentclass[fleqn,10pt]{SelfArx} % Document font size and equations flushed left

\usepackage[english]{babel} % Specify a different language here - english by default

\usepackage{marvosym, epigraph, subfig}

\usepackage[sortcites=false,style=authoryear-comp,bibencoding=utf8, natbib=true, firstinits=true, maxcitenames=2, maxbibnames = 99, uniquename=false, backend=bibtex, useprefix=true, backref=false,doi=false,isbn=false,url=false,dashed=true]{biblatex}
\setlength\bibhang{20pt}
\bibliography{ThomasReferenties.bib}
\AtEveryBibitem{%
	\clearfield{day}%
	\clearfield{month}%
	\clearfield{endday}%
	\clearfield{endmonth}%
}

%----------------------------------------------------------------------------------------
%	COLUMNS
%----------------------------------------------------------------------------------------

\setlength{\columnsep}{0.55cm} % Distance between the two columns of text
\setlength{\fboxrule}{0.75pt} % Width of the border around the abstract

%----------------------------------------------------------------------------------------
%	COLORS
%----------------------------------------------------------------------------------------

\definecolor{color1}{RGB}{0,0,90} % Color of the article title and sections
\definecolor{color2}{RGB}{0,20,20} % Color of the boxes behind the abstract and headings

%----------------------------------------------------------------------------------------
%	HYPERLINKS
%----------------------------------------------------------------------------------------

\usepackage{hyperref} % Required for hyperlinks
\hypersetup{hidelinks,colorlinks,breaklinks=true,urlcolor=color2,citecolor=color1,linkcolor=color1,bookmarksopen=false,pdftitle={What drives which region?},pdfauthor={Thomas de Graaff}}

%----------------------------------------------------------------------------------------
%	ARTICLE INFORMATION
%----------------------------------------------------------------------------------------

\JournalInfo{Statistical Software Applications} % Journal information 
\Archive{Internal note} % Additional notes (e.g. copyright, DOI, review/research article)

\PaperTitle{Which statistical software to use?}
\SubPaperTitle{A reflection and review upon the use of STATA, R and Python for
  teaching in the social sciences}

\Authors{Thomas de Graaff\textsuperscript{1}*} % Authors
\affiliation{\textsuperscript{1}\textit{Department of Spatial Economics, Vrije Universiteit Amsterdam, Amsterdam, The Netherlands}} % Author affiliation
\affiliation{*\textbf{Corresponding author}: \Letter{} t.de.graaff@vu.nl; \Mundus{} \href{thomasdegraaff.nl}{thomasdegraaff.nl}} % Corresponding author

\Keywords{Teaching---Data---Empirical research---STATA ---R---Python}
\newcommand{\keywordname}{Keywords} 

%%----------------------------------------------------------------------------------------
%%	ABSTRACT
%%----------------------------------------------------------------------------------------

\Abstract{}

%----------------------------------------------------------------------------------------

\begin{document}

\flushbottom % Makes all text pages the same height
\maketitle % Print the title and abstract box
%\tableofcontents % Print the contents section
\thispagestyle{empty} % Removes page numbering from the first page

%----------------------------------------------------------------------------------------

\section*{Introduction: the empirical workflow} % The \section*{} command stops section numbering

\epigraph{Econometrics is much easier without the data.}{Marno Verbeek}

The quote above does not only apply to economics and econometrics, but to all of
the social sciences as well. Empirical research---that is, dealing with data in
all its forms---requires a rigorous approach. Even more so, with the increasing
emphasis on openness and reproducability of research. Therefore, it is strange
that in academic education there is not much guidance in choosing which tools to
use and in the philosophy behing choosing an efficient and reproducable
workflow.\footnote{There are some exceptions, see, e.g.,
  \citet{healy2011choosing}.}

This note deals with the suitability of various software packages for applying
applied econometrics in specific and data science in general.\footnote{There
  is a difference between econometrics on the one hand and statistics on the
  other hand. Economics students first and foremost need to able to apply
  applied econometric techniques, such as presented in
  \citet{angrist2008mostly}}.
Specifically, In will focus on STATA, R and Python.\footnote{I will also briefly
  tough upon some other packages, but these three mentioned are most likely the
  most used in economics, except of course for the ubiquitous Excel.}
I will not say which tool to use. Instead, I will focus on the various strengths
and weaknessess of each software package combined with it specific approach. The
main criteria I will consider are the package suitability for education and how
well it can be integrated in an efficient workflow. The former is mostly
important for doing (small) data exercises, whilst the latter is vital for
larger research project, such as theses and later on perhaps research papers.  

To illustrate why the importance of reproducability, note that a typical empirical workflow in the social sciences looks as follows:
\begin{description}
\item[Generate data] Data is read from an external source (file or online database) or is simulated.  
\item[Manipulating data] This is usually the most time demanding
  phase\footnote{There is and old saying that says that 80\% of your research time
  goes in transforming data, while 20\% is only spent on analysing the data} and
  includes (amongst many other things) manipulating missing data, merging data
  and relabeling data
\item[Analyse data] This phase includes not only standard ecometrics and
  statistical or machine learning techniques, but as well as graphical
  representations as maps and figures. 
\item[Present results] Finally, documents in the forms of papers, posters, theses, or
  presentations have to be drafted. Note that ideally one wants to do in various
  formats, such as in pdf for physical paper and in html for webdisplay. 
\end{description}
Unfortunately, all these steps do not necessarily run sequentially. Supervisors,
referees, colleagues, and the future you, always want to add or delete elements
to or from your research. For instance, variables have to be added,
models specifications have to be checked, and 3D pie charts have to be changed
in something useful.

Therefore, it is vital that all these steps are both (\emph{i}) very well
documented so that the future you can easily retrace your steps, implement
changes and redo the whole research if needed, and (\emph{ii}) well connected to
each other. The latter does not necessarily entail that the whole research
should be done in one software environment, but instead that the outcome of one
research step (e.g., generating data) can easily serve as an input for another
step (e.g., data manipulation).

In the next section I will lay out the strengths and weaknessess of the three
statistical software packages according to the criteria mentioned above. 
\section*{Statistical software packages}

I review the various packages according to several criteria. There are several
other that will be discussed, but these I find most important for a suitable
software package to be used for teaching. 
\begin{description}
\item[Open source] The most important argument to use an open source package is
  reproducability. Your work is simply less accessible and thus reproducable if
  the code can only be run with applications that costs over \EUR{1,000}. 
\item[Learning curve] First and foremost, students should be able to use the
  package for straightforward econometric research. If that is not possible
  after one six-week's course, the software package is not particularly
  suitable. 
\item[Size of the community] Nobody want to be locked in with obsolete
  technology. A large userbase ensures a high probability that the software
  package will be used and maintaned in the future as well. Moreover, all sorts
  of indirect effects, such as user written routines, packages and
  documentation, come along for free with a large community. 
\item[Usefulness outside academia] Often forgotten as an argument but outside
  acadamic life, some applications are more used than others. And with the
  recent emphasis on better preparation for the labor market, this argument
  seems to become more importnant. By the way, the
  application still mostly used would be the ubiquitous Excel and its related
  visual basic scripts.
\item[Flexibility]
\end{description}

\subsection*{STATA}

\subsection*{R}

\subsection*{Python}

\section*{Statistics in education}

\section*{Concluding remarks}

%----------------------------------------------------------------------------------------
%	REFERENCE LIST
%----------------------------------------------------------------------------------------

\addcontentsline{toc}{section}{References} % Adds this section to the table of contents
\printbibliography

%----------------------------------------------------------------------------------------

\end{document}